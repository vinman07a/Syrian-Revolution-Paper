\documentclass[a4paper,titlepage,12pt]{turabian-researchpaper}

\usepackage[T1]{fontenc}
\usepackage[utf8]{inputenc}
\usepackage{lmodern}

\usepackage[english]{babel}
\usepackage{csquotes}

\usepackage{setspace}
\doublespacing

\usepackage{indentfirst}

\usepackage[notes,backend=biber]{biblatex-chicago}
\bibliography{sample}

\newgeometry{tmargin=1in, bmargin=1in, rmargin=1in, lmargin=1in}

\begin{document}
\title{Five Years After The Arab Spring: The Current State Of The Syrian
	Revolution, The Conflict That Grew Into The Worst Crisis In
	The World}
\author{Vincent Insinga}

\maketitle

\section{Introduction}

470,000 - At the time of this writing, nearly half of a million men, women, and
children have lost their lives as the direct result of a bloody war for a
country the size of Washington state. \autocite{death}
This conflict, which started with peaceful protests against the Assad regime in
March of 2011 and quickly developed into full revolution by the
end of the same year, is now one of the most divided conflicts in recent
history. Rebel groups were formed and organized along racial, religious, and ideological
lines; each fighting the other, the regime, and the self proclaimed Islamic
State in Iraq and Syria (ISIS). \autocite[2-5]{sectarian} The war in
Syria is a complex, yet profoundly urgent, emergency which  has the potential to decide the fate of
the entire region including the Kurdish quasi-state of  Rojava,
cause harm internationally, and defeat or accelerate an extremely powerful
militant Islamic group, which has branded itself as a state; additionally, the
war is a very large humanitarian issue for civilians within Syria.

\section{Background: History And Dynamics}

Understanding the history of this conflict is required in order to understand
the current state of the war. Syria's government was replaced most recently in
1970 in a coup by Bashar al-Assad's father, Hafez al-Assad. Assad installed an
autocratic and tyrannical dictatorship in the country. Bashar al-Assad
inherited the country after his father's death, he was initially quite popular,
espcially among the youth. However, after some time, things began to return to
the way they once were. In 2011, the Arab Spring reached
Syria. Arabs, Kurds and other peoples took to the streets to protest for
democracy and other reforms. Similar protests
had resulted in the removal of the leaders of Tunisia and Egypt not long
before. The protests consisted of entirely peaceful demonstrations and other
non-violent uprisings. In March of 2011 however, president Bashar al-Assad
decided to end the protests with overwhelming military force. The protesters
responded and the peaceful demonstrations turned into violent revolution against the
regime. \autocite[1-3]{overview} In 2013, two years after the fighting broke
out, ISIS began to gain ground in the region. That same year, the regime of
Bashar al-Assad became the only entity in the country that could rival the
human rights violations of the Islamic State when it committed a crime against
humanity by using chemical weapons on civilians in Ghouta.
\autocite{revolution}

It is important to note the large amount of different groups and factions which have
gotten involved in this conflict since these initial events transpired. The
Assad regime still stands, but the opposition forces have remained extremely fragmented.
Different rebel groups and factions are divided across every line, there are
Islamic radicals, even these are engaged in a sectarian war with each other,
and other groups organized on various sectarian divides, including
ideology and race. Even Kurdish autonomous region of
Rojava, one of the largest secular groups of rebels, were
initially organized on the basis of the Kurdish ethnic sub-state identity.
\autocite[3-4]{sectarian} Rojava has since become a society built on utopian
idealism, embracing direct democracy, libertarian socialism, and confederalism.
\autocite{rojavarev} The Syrian National Coalition for Syrian Revolutionary and
Opposition Forces, another opposition group, was formally recognized ``as the legitimate representative of the
Syrian people'' by the United Nations, on May 15 2013, after strongly
condemning ``the continued escalation in the use by the Syrian
authorities of heavy weapons, including indiscriminate shelling from tanks and
aircraft, and the use of ballistic missiles and other indiscriminate weapons against
population centres, as well as the use of cluster munitions'' and all other
violations of international humanitarian law taken by the government. \autocite{un1}
Additionally to these three, there are many islamic factions, these islamic
factions are no less divided than the other rebel factions. One coalition of
non-jihadist Sunni Islamist rebel groups is the Islamic Front.
\autocite[15]{sectarian} The Jihadist groups pose a much greater threat
to stability in the region, human rights within the country, and the
international community. Those in this group which are of particular threat are
the Islamic State in Iraq and Syria, and an Al-Qaeda affiliate called Jabhat
al-Nusra. \autocite[2]{isis} \autocite{nusra}

Each of these parties is backed by and intertwined with various international groups
and external forces. The government is ``backed by Russia, Iran, and the
Lebanese Shia Muslim political party and militant group Hezbollah'' \autocite{cfr}. The
Syrian National coalition has the, somewhat useless, backing of the United
Nations and legitimacy on the world stage. \autocite{legit} The Islamic Front
has support from, and was created by, Saudi Arabia. \autocite[15]{sectarian}
The Democratic Union Party (PYD), which is the most powerful political party in
Rojave, has ties to the Kurdistan Worker's Party (PKK). The PKK itself is a party
which is engaged in ``a low-intensity war'' against the Turkish state. \autocite[1-2]{defense} Despite
Turkish involvement in the fight agains ISIS and the Assad regime, both of
which are entities to which Rojava is opposed, Turkey is also violently opposed
to Kurdish independence in any form, including the autonomy of Rojava in Syria.
``Mr. Erdogan [the current president of Turkey] has said that Turkey `will
never allow the establishment of a new state on our southern frontier in the
north of Syria' [speaking of Kurdistan as a whole, not just Rojava].'' To
further complicate the matter, the United States is allies with Turkey and has been
providing assistance to Kurdish forces. \autocite{turkey} Unfortunately, Jihadist groups have
also seen outside support; ISIS has recruited more than 25,000 foreign
fighters \autocite{cfr}, while the al-Nusra Front is funded by al-Qaeda. ISIS, however, has
become the primary adversary of nearly every faction. \autocite{untangle}

In this way, the conflict is very complex; it is a revolution against an
oppressive regime; it is a sectarian conflict between different kinds of
radicals as well as more moderate groups; it is a proxy war between
international imperialist powers and powerful nations in the immediate area;
it is a global struggle against radical Jihad and the rising Islamic State.
Each group is involved with multiple sides of the war, the United States, for
example, is working together with nearly every other entity in the war on ISIS,
it is also working with its ally, Turkey, against the regime - making for a
proxy war between rebels backed by the United States and Turkey and government
forces backed by Russia and Iran. Despite this partnership with Turkey, The
United States has chosen to back Kurdish fighters, putting it at odds with
Turkey. Each part of this war is interwoven with the next, making for a messy
situation for foreign powers that are entangled with allegiances to each other.
\autocite{untangle} \autocite{cfr}

\section{International Threats And Effects: The Rise of ISIS, Relations Between
	Global Powers, The Stability of The Middle East, The Rise of Kurdish
	Autonomy, And The Question of Democracy}

Any war with so many factions, and so much international intervention will have
a global impact. In addition to its grave humanitarian crisis (which will be
discussed later), the Syrian conflict has and will continue to have a large
effect on the world. There is the grim possibility of a long term caliphate
created by al-Nusra or ISIS; there is the gloomy prospect of the Assad regime
retaining power; there is, however, a glimmer of hope - there is the hope for
democracy delivered by the Kurds, the Syrian National Committee, and other
groups who are fighting for freedom, and not for oppression.

The oppressive government of Syria has managed to retain some control in
Syria. In 2015, Bashar al-Assad used are strikes to assist ISIS in attacking
rebels in the Aleppo, a city in northern Syria. \autocite{assad} Assad is
a ruthless and tyrannical man who will do anything to keep his power; he will
kill innocents, he will help an organization that has proven to be the only on
worse than his own; he will have peaceful protesters massacred in the streets.

Jihadist groups
being able retain control of the nation and build a stable
caliphate is the primary threat to the Syrian people, and world as a whole.
The Islamic State in particular is the greatest menace upon the world. ISIS is
recruiting the most fighters, making the most money, and claiming the most land
of any current terrorist organization. \autocite{isis} The other powerful
jihadist group in the region is the Jabhat al-Nusra Front. This rebel group,
backed by al-Qaeda, has been fighting along side, not against, moderate rebels
in Syria. ``Jabhat al-Nusra will use the legitimacy gained by fighting
alongside the opposition to transform Syrian society until it accepts al
Qaeda.'' Says Jennifer Cafarella, a member of the Institute For The Study of
War, where she focuses on the Syrian conflict; ``The group is creating
structures of governance, like courts and social services, and using them to
transform the religious views of Syrian opposition groups and populations.''
She claims that al-Nusra is a greater threat than ISIS. \autocite{nusra} Both
of these Jihadist groups are great threats which must be defeated.

On Thursday, February 25th, a Kurdish activist tweeted a video of a Kurdish
female fighter ripping down, and throwing to the ground, the ISIS flag in Shaddadi's main square and
replacing it with that of the Syrian Democratic Forces (SDF). \autocite{tweet}
When a guerrilla army is fighting against an entity that treats women and minorities
in the way that ISIS does, by enslaving, raping, and torturing them,
\autocite[1]{isis} it seems
only fitting that women should among those to depart them from this world.

The Syrian revolution has become a proxy war between the United States, Turkey,
and Saudi Arabia, backing the rebels, and Russia and Iran, backing the regime.
Additionally, Turkey and America are at odds; America is supporting the Kurdish
forces, while Turkey is violently opposed to them. The war has become, not
about stabilizing the region, but about the way in which it will be stabilized.
Who will the future government find itself indebted to? Will the nation be more
westernized than before? Will it be built upon western ideas like democracy?
Syria will eventually stabilize, the question that is yet to be answered
is that of the kind of influence that it will be on the region; will it be a
force of terror and oppression, or of democracy and freedom?

\section{The People of Syria: Diaspora And A Human Rights Catastrophe}

The other major question is that of human rights. At the time of this writing,
the Syrian conflict has resulted in more than 470,000 deaths \autocite{death}, 4,100,000 million registered
refugees, and 6,500,000 displacements. This is, for reference, two percent of
Syria's 22.85 million population dead, eighteen percent refugeed, and
twenty-eight percent displaced. The humanitarian disaster, however, expands
far beyond these grim numbers. Syrians are still suffering at the hands of the
regime, ISIS, the ongoing war.
\autocite{cfr}

Human rights violations hardly started with Bashar al-Assad, his father was
known for oppression and tyranny. 
\begin{quote}The state of emergency, which was in effect
between 1963 and 2011, granted the government sweeping powers of arbitrary
detention and arrest. The government strictly restricted the freedom of
movement, expression, and organization. The single party regime under the
hegemony of the Baath Party and Assad’s Family carried out gross human rights
abuses.
\end{quote} 
Hafez al-Assad massacred thousands of people at a time in the 80s.  Bashar
al-Assad continued on the path of his father, his regime, still today, is
abusing and torturing prisoners, executing people without trial, and
systematically using police power for oppression through``mass arrests,
torture, rape, forcible displacement,
abductions, forced disappearances,
pillaging and destruction of property,
degrading or inhumane treatment.''

As always, it seems that there is one group which is capable of rivaling the
Assad regime's monstrosity. However in the case of human rights violations,
that is not so. ISIS appears to be the grater killer because of the way that it
propagandizes it's violence. The regime, on the other hand, ``kills Syrians by
air attacks and shelling of civilian areas,
blocking humanitarian aid to the
opposition-controlled areas or killing
the citizens of Syria in detention centers
and prisons as a result of torture.'' It does these things in a much more
hidden way than ISIS conducts its public beheadings. \autocite{human}

Despite not being as murderous as the Syrian government, ISIS is still an awful
scourge on the people of Syria. The Islamic State ``continues to attack schools
and hospitals, kill civilians, and displace minorities.'' \autocite[20]{isis}
This organization has made and internet recruitment campaign out of showcasing
their brutality on social media. \autocite[8]{isis} In a similar manner to Nazi
Germany, ISIS commits genocide on racial and religious minorities. They keep
order through public executions and beatings. \autocite[2-3]{isis}

\section{In Conclusion: Looking Forward}

The Syrian conflict is the most pressing crisis of current times, the outcome
of this war effects greatly the people within Syria, the immediate region which
is threatened by ISIS, Kurdistan, Turkey, the entirety of the middle east, and
the whole of the world; the war is a humanitarian crisis, a tipping point for
democracy, and a breeding ground for violence. There is no question who is evil
in this war. There is no question who is on the wrong side of this conflict.
There is no question that those who indiscriminately murder civilians are the
adversaries of humanity. There is no question that Assad regime and the Islamic
State are fighting for the wrong reasons: for oppression, for terror, for
injustice, they are fighting for evil itself. Just as there is no question that
these groups' reasons for fighting are wrong, there is no question that the
only way to defeat those fighting for the wrong things and the wrong reasons is
to support those fighting for the right reasons. The only way forward is to
support the Kurdish forces, the National Coalition for Syrian Revolutionary and
Opposition Forces, and other groups who are fighting for freedom. The only way
to ensure that barbarians and monsters never again rule Syria and oppress the
Syrian people is to put the people in charge of the nation and institute
democracy - not by force and coercion from the west - but in a truly democratic
manner, by giving the people control of their own destiny.



\printbibliography

\end{document}
